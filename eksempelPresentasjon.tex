\documentclass{beamer}

% ------------------------------------------------
% Bruk temaet direkte fra repoet (uten MiKTeX-oppsett)
% Repo-struktur (fra denne fila):
%   tex/latex/beamer/themes/FSV-style/beamerthemeFSV.sty
%   tex/latex/beamer/themes/FSV-style/fsv_logo.jpg
% ------------------------------------------------
\makeatletter
\def\input@path{{tex/latex/beamer/themes/FSV-style/}}
\makeatother
\usepackage{beamerthemeFSV} % finner .sty via input@path

% Kun nødvendige pakker for innhold i presentasjonen


% Metadata
\title{Eksempelpresentasjon}
\subtitle{Demo av FSV Beamer Theme (plug-and-play)}
\author{Forfatter Navn}
\newcommand{\authshort}{FN}
\institute{Fagskolen Viken}
\date{\today}

\begin{document}
	
	\begin{frame}
		\titlepage
	\end{frame}
	
	\begin{frame}{Hvordan bruke temaet}
		\begin{itemize}
			\item Klon repoet.
			\item Kompiler denne fila direkte — ingen MiKTeX-oppsett nødvendig.
			\item Alternativt: legg repo-roten som TEXMF-root i MiKTeX og bruk \texttt{\textbackslash usetheme\{FSV\}}.
		\end{itemize}
	\end{frame}
	
	\begin{frame}{Eksempel: Tekst}
		\begin{itemize}
			\item Punkt 1
			\item Punkt 2
		\end{itemize}
	\end{frame}
	
	\begin{frame}{Eksempel: Tekst og bilde}
		\begin{columns}
			\column{0.5\textwidth}
			\begin{itemize}
				\item Tekst til venstre
				\item Beskrivelse av bildet
			\end{itemize}
			\column{0.5\textwidth}
			\includegraphics[width=\linewidth]{example-image}
		\end{columns}
	\end{frame}
	
	\begin{frame}{Anbefalt PDF-viser}
		\begin{itemize}
			\item Bruk \textbf{Pympress} for presentasjonsmodus med notater.
			\item \url{https://pympress.github.io/}
		\end{itemize}
	\end{frame}
	
\end{document}
